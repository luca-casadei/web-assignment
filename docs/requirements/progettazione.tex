\documentclass[a4paper]{report}
\usepackage[utf8]{inputenc}
\usepackage[italian]{babel}
\usepackage{hyperref}
\usepackage[left=1.5cm, right=1.5cm, bottom=1.5cm]{geometry}
\pagestyle{empty}

\author{Luca Casadei, Franscesco Pazzaglia, Andrea De Carli}
\title{\textbf{Design e progettazione\\Elaborato Web 2024}}
\date{Ultima modifica: \today}

\begin{document}
	\maketitle
	\section{Colore}
	La prima fase consiste nella realizzazione del design UI dell'applicazione, per cominciare, abbiamo scelto una palette di colori con sufficiente contrasto da essere in linea con la specifica \textit{WCAG AAA}, in particolare, sono stati scelti i colori in codifica esadecimale: \textit{801524}, \textit{D9D9D9} e \textit{FFFFFF}.
	\section{Design mobile-first}
	Per poter utilizzare l'approccio di \textit{progressive enhancement}, è stato necessario per prima cosa realizzare il design adatto a dispositivi mobili; come dispositivo di riferimento, infatti, è stato usato uno dei modelli con la dimensione dello schermo più piccola, l'\textit{Iphone SE}.
	\subsection{Pagina Home} \label{ss:home}
	Come prima pagina appena si digita l'URL del sito, si ha la \textit{HomePage}, sulla quale si può visualizzare per prima cosa l'immagine della copertina del sito e alcuni post di vendita nello spazio immediatamente successivo. Nella parte superiore della Home si vede la barra di navigazione, su cui sono presenti il logo, il pulsante di notifiche, quello del profilo e, infine, il pulsante di navigazione che, se aperto, mostra il menu dal quale si può navigare nella pagina di ricerca avanzata o nel carrello.
	\subsection{Pagina di ricerca}
	Navigando alla pagina di ricerca avanzata, viene presentata una pagina simile alla Home, con la possibilità di ricercare e filtrare gli annunci di vendita di un libro basandosi su categoria, genere, autore, e prezzo, e di ordinare i risultati in ordine alfabetico, prezzo più alto ecc\dots
	\subsection{Pagina di registrazione e di accesso}
	Se si tenta di andare nella pagina del carrello \hyperref[ss:home]{(vedi Home)} o di cliccare sul bottone di aggiunta al carrello \hyperref[ss:home]{(vedi dettaglio vendita)} senza aver fatto l'accesso in precedenza, si verrà portati alla pagina di login, la quale presenta i campi necessari per poter effettuare l'accesso. Qualora si non avesse ancora un account, è possibile registrarsi cliccando sul link situato in fondo alla pagina. I dati personali di un utente possono essere visualizzati e cambiati dalla pagina dell'account \hyperref[ss:home]{(vedi Home)}.
	\subsection{Pagina di dettaglio di vendita} \label{ss:dettvend}
	Per poter aggiungere un libro al carrello, è necessario cliccare su un articolo presente in lista. Dopodiché, vengono presentate le informazioni relative al prodotto come la descrizione del libro, le immagini associate all'annuncio di vendita, l'autore, il genere, la categoria ed l'ISBN. Per inserire il libro nel carrello sarà sufficiente cliccare sul relativo bottone.
	\subsection{Pagina di carrello}
	In questa pagina, tutti i libri aggiunti in precedenza vengono raggruppati per poter essere acquistati in una volta sola. Compaiono quindi il prezzo complessivo e il bottone per poter procedere all'ordine e al pagamento. E' possibile inoltre rimuovere articoli dal carrello cliccando sull'apposita \textit{X}.
	\subsection{Pagamento e ordine}
	Cliccando sul bottone \textit{"Procedi all'ordine"} nella pagina del carrello, si verrà indirizzati alla pagina in cui è possibile inserire i propri dati di pagamento: se il tutto avviene con successo, i libri acquistati potranno essere visualizzati nella pagina degli ordini, nella quale è possibile anche monitorare lo stato di consegna relativo a ciascun ordine.
	\subsection{Pagina del venditore}
	Quando l'utente venditore effettua il login, viene reindirizzato su una parte del sito differente, alla quale l'utente cliente non ha accesso. All'interno di questa pagina è possibile registrare libri o modificarne di già esistenti, aggiungere annunci, visualizzare gli ordini degli utenti e cambiarne lo stato.
\end{document}