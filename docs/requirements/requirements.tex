\documentclass[a4paper]{article}

\usepackage[italian]{babel}
\usepackage[utf8]{inputenc}
\usepackage{hyperref}
\usepackage{float}
\usepackage{todonotes}


\author{Luca Casadei\\Francesco Pazzaglia\\Andrea De Carli}
\date{Ultima modifica: \today}
\title{\textbf{Specifiche progetto Tecnologie Web}}

\begin{document}
	\maketitle
	\tableofcontents
	\section{Testo fornito}
	"Scrivere una applicazione web accessibile e responsive che consenta di comprare dei generici prodotti (es:	cibo, beni di prima necessità o qualcosa di più fantasioso) con consegna da effettuare presso il nostro Campus.\\
	Devono esserci:
	\begin{itemize}
		\item Utente Venditore che gestisce un listino dei prodotti e gli ordini ricevuti.
		\item Utenti Clienti che possono registrarsi al sito e acquistare uno o più prodotti.
	\end{itemize}
	Possono esserci funzioni aggiuntive che faranno parte del design\dots\quotedblbase
	\section{Adattamento del testo}
	L'applicazione consente di comprare e mettere in vendita libri (nuovi o usati) da parte di studenti o persone esterne, è prevista quindi una parte di utenza interessata all'acquisto di libri (cliente) che potrà effettuare degli ordini, ed un'altra parte che metterà in vendita i propri libri (venditore). La consegna di questi ultimi avviene direttamente al Campus dell'università di Bologna situato a Cesena. A seguito della corretta effettuazione di tutto il processo di ordinazione, che prevederà l'inserimento dei dati di pagamento, verrà inviata una notifica al venditore.\\
	\todo{FINIRE}
\end{document}